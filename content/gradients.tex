\documentclass{article}
\usepackage{fancyhdr}
\usepackage[a4paper, total={7in, 10in}]{geometry}
\usepackage{enumitem}
\usepackage{mathbbol}
\usepackage{xcolor}
\usepackage{forest}
\usepackage{soul}
\usepackage{graphicx}
\usepackage{amsmath}
\usepackage{setspace}
\usepackage{listings}
\usepackage{color}
\pagestyle{fancy}
\onehalfspacing
\lhead{David Saldubehere}
\author{David Saldubehere}
\begin{document}
\section{Gradient Descent Tutorial}
You may be familiar with functions in 2D space.

For example, we can have \(f(x) = x^2\)

PLOT THIS

We could add a dimension and have $f(x, y) = x^2 + y^2$

PLOT THIS

Where $f(x, y)$ is the function that takes in a point in 2D space and returns a value.

Some people label $f(x, y)$ as z

Recall what the derivative of $f(x)$ is. It is the slope of the tangent line at a point which is the rate of change of $f(x)$ with respect to $x$ (how does $f(x)$ change as $x$ changes) 

How do we format this in the context of $f(x, y)$?

So now we have two questions to answer, how does $f(x, y)$ change as $x$ changes and how does $f(x, y)$ change as $y$ changes?

We can find this by taking the partial derivative of $f(x, y)$ with respect to $x$ and $y$.

The partial derivative of $f(x, y)$ with respect to $x$ is denoted as $\frac{\partial f}{\partial x}$ and the partial derivative of $f(x, y)$ with respect to $y$ is denoted as $\frac{\partial f}{\partial y}$.

For the previous example, $f(x, y) = x^2 + y^2$, we have $\frac{\partial f}{\partial x} = 2x$ and $\frac{\partial f}{\partial y} = 2y$.

We simply take the derivative of $f(x, y)$ with respect to $x$ and $y$ and treat the other variable as a constant.

$\frac{\partial f}{\partial x} = 2x$ means that as $x$ changes, $f(x, y)$ changes at a rate of $2x$.

The gradient of $f(x, y)$ is the vector (a vector is an object with magnitude (size) and direction )of the partial derivatives of $f(x, y)$ with respect to $x$ and $y$. It is denoted as $\nabla f(x, y)$.

Lets do another example, $f(x, y) = -x^2 -y^2 + 10$

$\frac{\partial f}{\partial x} = -2x$ and $\frac{\partial f}{\partial y} = -2y$

So the gradient of $f(x, y)$ is $\nabla f(x, y) = \begin{bmatrix} -2x \\ -2y \end{bmatrix}$

Lets plot this function

PLOT THIS

The gradient of $f(x, y)$ at a point $(x, y)$ is the vector corresponding to the direction that would increase $f(x, y)$ the most at that point.

Key points:

The gradient of $f(x, y)$ at a point $(x, y)$ is the vector corresponding to the direction that would increase $f(x, y)$ the most at that point.

A gradient is simply a vector of partial derivatives.

A partial derivative is the rate of change of a function with respect to one of its variables.


\end{document}